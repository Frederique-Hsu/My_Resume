% File name     : Work_Experience.tex
% Description   : It places the "Work_Experience" CVSection in this script
% Author        : Frederique Hsu
% Date          : Sun.  13 Oct. 2024
%
%


\documentclass[./CV]{subfiles}


\begin{document}
    \engcvsection{\faGears{} Experience}
    
    \engcvsubsection{Jul. 2023 - May. 2024}
                    {\makecell[l]{C++ dev. expert\\(CAD geometry constraint solver engine)}}
                    {PoissonSoft Shanghai}
                    
    \cvdetail{
        Participated in the development task about the constraint solver engine of domestic 3D CAD industrial design software,
        which has 2 core engines: Geometry Modeling, Geometric Constraint Solving. The GCS engine is used to efficiently solve the large
        scale of nonlinear equations, have the constraint conditions satisfied among various geometry elements, when engineer draws/designs
        with CAD software.
        
        \faCaretRight{}\quad{} I had helped to design a hierarchical, message-passing based, and easy-to-horizontal-scale dynamically software architecture, according
        to the amount of connected graph components. This engine is programmed in C++17, consisted of such hierarchy layers as below:
        \begin{center}
            \begin{tabular}{cl}
                1 & User API layer \\
                \hline
                2 & Geometry element(a.k.a. vertex) - constraint relationship(a.k.a. edge) graph model layer \\
                \hline
                3 & Geometric decomposition layer according to connected component \\
                \hline
                4 & Layer of converting constraint graph to nonlinear equations \\
                \hline
                5 & Layer of jacobianizing nonlinear equations to solve, based on Newton-Raphson iteration method \\
            \end{tabular}
        \end{center}
        \faCaretRight{}\quad{} Comparing with the benchmark Siemens D-Cubed engine in CAD industry, this engine has offered as rich features as 175+ interfaces, plus the 
        toolsets to visualize the solving process and assess the effect. In the actual testing, when fully solving such a big nonlinear equations with
        2M geometry elements and 30K constraints, it just costs less than 5 sec, 87\% of D-Cubed performance. We have created the top performance index
         of domestic CAD GCS engine. 
    }
    
    \cvdetail{
        Independently developed a suite of Sketcher visualization tool, by virtue of Qt Widget framework, and the CAD.3DViewer lite, they have 
        integrated in the PCSC Engine, to validate the functionalities of engine inside our team, meanwhile test the actual solving effect and performance.
        
        \faCaretRight{}\quad{} Additionally, took charge of the automatic CI building and Regression Test Pipeline, push the agile development with the optimized workflow: 
        GitLab + Jira + ReviewBoard + CTest/CPack/CDash for all team players.
    }
    
    \cvdetail{
        Applied a patent: A kind of parallel accelerating method, applying the OpenGL hybrid heterogeneous computing on geometry constraint solving system.
        
        \faCaretRight{}\quad{} This patent takes good use of OpenCL parallel heterogeneous computing framework for the 1st time, utilizing such computing resources as GPU. FPGA
        and DSP to accelerate the super-large-scale Jacobian matrix decomposition for geometry constraint solving system. 
        
        \faCaretRight{}\quad{} Applying this patent had dramatically
        improved the performance, decreasing the complexity of matrix multiplication from Strassen algorithm $O(n^{2.48})$ down to current $O(n^2 + n\log_{2}n)$.
    }
    
    \vspace{10pt}
    \textcolor{blue!30}{\hrule}
    \engcvsubsection{Mar. 2022 - Feb. 2023}
                    {\makecell[l]{C++ developer\\(Autonomous driving firmware system)}}
                    {\makecell[r]{BMW (Shanghai) Auto-\\nomous Driving Lab.}}
    
    \cvdetail{
        With my team to develop the Road Model module of L2 autonomous driving for BMW 3series, i.e. the experiment car gathering the driving trajectory data
        over such sensors as Lidar, mmRadar, Camera array, etc., then parse and fusion out the virtual lanes, relative to the physical lanes. They were 
        submitted to the decision system, finally guide the EgoVehicle to drive along the virtual lanes. Visualize the trajectories, virtual lanes and video 
        stream on the ROS RViz emulator, to validate and test the Road Model.
        
        \faCaretRight{}\quad{} We have implemented the Road Model using C++ and Python on Ubuntu Linux platform, auto-built the software with Bazel build system, wrote many test 
        scenarios in GoogleTest framework, orchestrated the Docker container cluster to make the full-scenarios testing for each feature.
        
        \faCaretRight{}\quad{} After these test items, the firmware was programmed into domain controller, and executed the HIL, SIL testing, finally driver/vehicle-in-loop experiment.
    }
    
    \cvdetail{
        Actually I had participated or took responsible of these sub-modules of Road Model module as below:
        
        \begin{center}
            \begin{tabular}{cl|cl|cl}
                1 & RoadTrackerModule   & 2 & TrajectoryTrackerModule & 3 & ClusteringModule  \\
                \hline
                4 & CentralFusionModule & 5 & LaneFinderModule        & 6 & LaneTrackerModule \\
            \end{tabular}
        \end{center}    
        
        These sub-modules were executed like a pipeline, it is exactly the software architecture we've designed.
        
        \begin{tabular}{l}
            \faCaretRight\quad{}\underline{RoadTrackerModule} computes out the curved shape and boundary of road.\\
            
            \makecell[l]{\faCaretRight\quad{}\underline{TrajectoryTrackerModule} calcuates the movement trajectory of each vehicle by using the Kalmann Filter 
                \\\quad{} algorithm, according to the position, direction and status of surrounding vehicles, which are relative to 
                \\\quad{} EgoVehicle, at every time slot. What's more, save these trajectories and update them.}\\
            
            \makecell[l]{\faCaretRight\quad{}\underline{ClusteringModule} adopts the Kruskal/Prim algorithms to classify the trajectories and aggregate 
                \\\quad{} into Left/Ego/Right 3 bunches.}\\
        \end{tabular}
    }
    
\end{document}